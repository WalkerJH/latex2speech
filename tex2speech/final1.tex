\documentclass[letterpaper,12pt]{article}
\usepackage{graphicx}
\nonstopmode

Start of comment  Document Information
\title{Tex2Speech\\Vision and Scope}
\author{Connor Barlow, Walker Herring, Jacob Nemeth,\\Dylon Rajah and Taichen Rose}
\date{October 23, 2020}

Start of comment  Packages 
\usepackage[margin=1in,letterpaper]{geometry}
\usepackage{tabularx}



\begin{document}
\maketitle

\begin{center}
  Version: 1.1
\end{center}

\vfill
\begin{tabularx}{\linewidth}{|l|l|l|X|}\hline
Ver. & Date & Who & Change\\\hline
1.1  & 10/23/20  & Connor Barlow  & Added client considerations in risks and limitations\\\hline
\end{tabularx}
\newpage

background.tex Input file not found 
\section{Business Requirements}
\subsection{Background}
% What is the business context for the project?
% What does the reader need to know about the customer's industry?
For every working researcher, staying up to date with all the newest developments in their field of study is crucial. Since the goal of research is to build upon previous discoveries to make new ones, the first step is to know what has already been discovered. This is for the most part accomplished by reading papers published by other researchers in the same field and must be done continuously. Therefore, increasing the possible times to read papers, and in general increasing their accessibility is important enough to warrant the development of new tools to aid in this.

\subsection{Business Opportunity}
% What is the customer's need?
% How does this need fit in the industrial context?
Our customer has found that time spent driving could be used as a time to read additional papers if they could be read aloud. As the popularity of audiobooks, podcasts, and other spoken media shows, this is not an uncommon need at all and extends beyond driving to simply any time where our minds are idle but our hands are occupied. In the context of research, extending the ability to read papers to these occasions would be a boon for any working researcher.

\subsection{Business Objectives and Success Criteria}
% Why does the customer need this product?
% How will the customer know that the product is a success?
%   You must be very specific here.  What experiment can we do to verify success?
\par
Unfortunately, unlike more common forms of text media, scientific papers have no real good option for being read aloud. Traditional text-to-speech (TTS) systems have made most traditional texts available as audio, but the complexities found in most technical writing translate poorly (or not at all in the case of mathematical equations) to an audio equivalent. Other products address this problem, but they have a limited feature set that makes reading an entire paper from top-to-bottom difficult if not impossible.\\
\par
\noindent Our solution is \TeX 2Speech, a program that will parse the most common language for technical writing, \LaTeX\, into a form that can be read appropriately by existing TTS systems. To succeed, \TeX 2Speech will need to parse any arbitrary \LaTeX\ file and give the user a fully rendered audio output of the document as a result. It will have to remove marks not relevant for an understanding of the text, while also being able to indicate proper phrasing of the text, especially mathematical expressions that existing TTS systems have trouble with. To increase accessibility, the project will be available as a service over the internet, and users will interact with \TeX 2Speech by uploading their \LaTeX\ files and downloading the resulting audio.

\subsection{Customer and Market Needs}
% Connect the objectives and success criteria back to the customer's business.
% That is, why does the industry require the customer to have this solution? 
\TeX 2Speech would provide the solution to our customer’s desire for reading papers while otherwise indisposed. As was noted before, any working researcher needs to be continuously up to date on the latest relevant papers, and our solution will provide an avenue to do so more efficiently. For those with long commutes, unavoidable delays, or who simply enjoy audio media in the many forms it already exists in, \TeX 2Speech will extend that option to technical writing in its many forms.

\subsection{Business Risks}
% This is a hazard assessment.  What could go wrong?  How bad would it be if it
% did?  How likely is it?  What steps can we take to protect against the hazard? 
If users try to open up a gigantic markup \LaTeX\ file, depending on the implementation of \TeX 2Speech this could crash the program. While most \LaTeX\ documents in our use case are reasonably sized, open-source textbooks make this a real if unlikely concern and the consequences could be severe (crashing, non-functional program). \\

\noindent \LaTeX\ is a very flexible language, so users could reasonably use a package or command we didn’t take into account. This would be a high probability risk, with at least a medium severity since every use of the command would have a chance to have our TTS system say something with no equivalent English meaning.\\

\noindent Regardless of what we are parsing, there is always the risk of mis-parsing or mis-pronouncing any number of symbols. While the likelyhood of this happening would depend on both the file being parsed and our implementation, the severity would either be low to medium depending on the importance of the given word or symbol being pronounced.\\

\noindent Finally, the third-party software we’re using might be shut down or not online for some time. This would be low probability since the odds of a third party shutting down usually only happens in maintenance or a problem on their ends. This would also be low severity for us since we can just go to another third party software. 


\section{Vision of the Solution}
Begin Comment  Now that we know what the customer needs, 
Begin Comment  what will the solution look like? 
\subsection{Vision Statement}
Begin Comment  This is the formal vision statement.  
Begin Comment 
Begin Comment  For:         user class
Begin Comment  Who:         statement of need
Begin Comment  The:         title of product
Begin Comment  Is:          statement of solution
Begin Comment  Unlike:      closest alternative solution
Begin Comment  Our Product: differentiation statement  
For on the go researchers and academics who wish to increase their available paper reading time, \TeX 2Speech will provide on-demand speech synthesis for papers and a wide variety of other scientific and technical documentation written in the \LaTeX\ format. Unlike standard TTS systems, \TeX 2Speech will be able to effectively parse a \LaTeX\ document containing mathematical equations, and convert it to comprehensible spoken word.

\subsection{Major Features}
Begin Comment  Describe the major features that solve the customer's needs.
Major features include the ability to take an arbitrary \LaTeX\ document and convert it into an audio representation of the text. Major commands and environments will be supported in this conversion to benefit the majority of users. The primary feature enabling this would be our implementation to convert \LaTeX\ files into Speech Synthesis Markup Language (SSML), or something similar. The marked-up file would be fed to a TTS synthesis program for the audio output. Features that will be visible to the user include the option to upload their \LaTeX\ document to our service, as well as downloading the resulting audio. 

\subsection{Assumptions and Dependencies}
Begin Comment  What does this product depend on?  
Begin Comment  The goal here is to describe, with as much detail as possible, what the envisioned
Begin Comment  solution needs to operate.  
It must be assumed that users have access to the internet to utilize this program. Our service will depend both on some web hosting location, along with some cloud-based TTS service, either one failing will render our service inoperable.
\section{Vision of the Solution}
% Now that we know what the customer needs, 
% what will the solution look like? 
\subsection{Vision Statement}
% This is the formal vision statement.  
%
% For:         user class
% Who:         statement of need
% The:         title of product
% Is:          statement of solution
% Unlike:      closest alternative solution
% Our Product: differentiation statement  
For on the go researchers and academics who wish to increase their available paper reading time, \TeX 2Speech will provide on-demand speech synthesis for papers and a wide variety of other scientific and technical documentation written in the \LaTeX\ format. Unlike standard TTS systems, \TeX 2Speech will be able to effectively parse a \LaTeX\ document containing mathematical equations, and convert it to comprehensible spoken word.

\subsection{Major Features}
% Describe the major features that solve the customer's needs.
Major features include the ability to take an arbitrary \LaTeX\ document and convert it into an audio representation of the text. Major commands and environments will be supported in this conversion to benefit the majority of users. The primary feature enabling this would be our implementation to convert \LaTeX\ files into Speech Synthesis Markup Language (SSML), or something similar. The marked-up file would be fed to a TTS synthesis program for the audio output. Features that will be visible to the user include the option to upload their \LaTeX\ document to our service, as well as downloading the resulting audio. 

\subsection{Assumptions and Dependencies}
% What does this product depend on?  
% The goal here is to describe, with as much detail as possible, what the envisioned
% solution needs to operate.  
It must be assumed that users have access to the internet to utilize this program. Our service will depend both on some web hosting location, along with some cloud-based TTS service, either one failing will render our service inoperable.

\section{Scope and Limitations}
Begin Comment  This section identifies what part of the envisioned solution is going to be
Begin Comment  developed in this current project or version, what will be delayed until
Begin Comment  latter projects, and what is out of scope for the whole product.  
Begin Comment 
\subsection{Scope of Initial Release}
The initial release of the program will demonstrate the functionality of synthesizing audio from generic \LaTeX\ documents, with support for major commands and environments and the ability to parse common mathematical expressions. This will be hosted on a basic web service with the capability to upload \LaTeX\ files and download the resulting audio.

\subsection{Scope of Subsequent Releases}
\par
Future releases would focus on increasing the scope of environments and tools supported by \TeX 2Speech, expanding the number of accepted file formats accepted (e.g. entire compressed \LaTeX\ projects), and providing a more robust user interface for convenient reading of rendered \LaTeX\ files. Including support for one or more other languages outside the locale of English would be an important possibility to consider as well.\\
\par
\noindent The most ambitious possibility for a future release would be adding functionality to reverse engineer the \LaTeX\ source from PDFs that were generated with \LaTeX\. No accessible tools to do this currently exist, but if accomplished would expand the scope of this project to the huge number of papers/documents/textbooks for which no \LaTeX\ source exists. Even without this ambitious feature, websites like arXiv.org provide not only an enormous repository of papers but typically include their source, which will give \TeX 2Speech a huge library of files to synthesize regardless.

\subsection{Limitations and Exclusions}
Since \LaTeX\ is such an extendable language, it’s an impossibility to assume we could fully cover all environments and commands. No matter what \TeX 2Speech will operate on a “best-effort” system for unknown commands or combinations thereof.
\section{Scope and Limitations}
% This section identifies what part of the envisioned solution is going to be
% developed in this current project or version, what will be delayed until
% latter projects, and what is out of scope for the whole product.  
%
\subsection{Scope of Initial Release}
The initial release of the program will demonstrate the functionality of synthesizing audio from generic \LaTeX\ documents, with support for major commands and environments and the ability to parse common mathematical expressions. This will be hosted on a basic web service with the capability to upload \LaTeX\ files and download the resulting audio.

\subsection{Scope of Subsequent Releases}
\par
Future releases would focus on increasing the scope of environments and tools supported by \TeX 2Speech, expanding the number of accepted file formats accepted (e.g. entire compressed \LaTeX\ projects), and providing a more robust user interface for convenient reading of rendered \LaTeX\ files. Including support for one or more other languages outside the locale of English would be an important possibility to consider as well.\\
\par
\noindent The most ambitious possibility for a future release would be adding functionality to reverse engineer the \LaTeX\ source from PDFs that were generated with \LaTeX\. No accessible tools to do this currently exist, but if accomplished would expand the scope of this project to the huge number of papers/documents/textbooks for which no \LaTeX\ source exists. Even without this ambitious feature, websites like arXiv.org provide not only an enormous repository of papers but typically include their source, which will give \TeX 2Speech a huge library of files to synthesize regardless.

\subsection{Limitations and Exclusions}
Since \LaTeX\ is such an extendable language, it’s an impossibility to assume we could fully cover all environments and commands. No matter what \TeX 2Speech will operate on a “best-effort” system for unknown commands or combinations thereof.

busness_context.tex Input file not found 

\section{Business Context}

\subsection{Stakeholder Profiles}
% This section describes who will benefit from the product both directly and
% indirectly.  Identify stake holders by their roll and describe their needs and
% interactions for that role.
\par
Researchers and academics would be able to benefit from \TeX 2Speech. Since researchers need to stay up to date with current findings, this requires time to read other research papers. With \TeX 2Speech, this can now be done by listening on the go, or while performing other tasks where one could not sit down and read such papers. This could save researchers time and allow them to focus on other aspects of their research.\\
\par
\noindent People with disabilities would benefit in several different ways with \TeX 2Speech. Users that have vision problems would have access to getting important work translated into an audio format. Other users who learn better or focus on audio than physically reading something would also find benefit.

\subsection{Project Priorities}
% Prioritize the major features identified above based on the relative
% return-on-investment.
\par
The project’s primary priority is to get the ability to add a \LaTeX\ file into a converter and get it to successfully read the major \LaTeX\ commands and environments. Once this has been completed, secondary priorities will include enhancing the commands and environments that the \LaTeX\ reader can comprehend, working with text analysis to improve pronunciation, and implementing a web application for users to upload their files.\\
\par
\noindent The primary priority was chosen because all secondary priorities will build upon it. Once the \LaTeX\ file can successfully get outputted in an audio response, project members will be able to build upon this framework to get an accessible and useful application. If the primary priority is not properly implemented, then project members will not be able to move forward with enhancements.

\subsection{Operating Environment}
% Describe how and where the product will be used.  
This program is expected to run online allowing any user, on a mobile or computer device, the ability to use the features regardless of operating system.
\section{Schedule}

\subsection{Gantt Chart}
Begin Comment  The initial task break-down schedule displayed graphically as a Gantt chart.
Begin Comment  (see https://www.fool.com/the-blueprint/gantt-chart/)
Begin Comment  Gnatt charts list the tasks, show the length of each task, and dependencies
Begin Comment  between tasks.  
\begin{figure}[h]
\includegraphics[width=14cm]{docs/vision_and_scope/Gantt.PNG}
\end{figure}

\subsection{Key Milestones}
Begin Comment  Milestones are points in the development where you have implemented some
Begin Comment  number of major features.  For senior project, you generally have five
Begin Comment  milestones: one at the end of 491, three in 492, and one in 493.
Begin Comment  In this section, divide the Major Features between the milestones.  You should
Begin Comment  front-load the schedule as much as possible.  That is, leave 10Begin Comment  of the work
Begin Comment  for the last milestone.
Milestone 1: Research and Implement Pipeline
\begin{itemize}
    \itemsep0em
    \item Research software to use for the project
    \item Setup TTS Web Service
    \item Find a placeholder for \LaTeX\ to SSML/MathML
    \item Find a parsing service for \LaTeX\ file
    \item Create simple web service for uploading files
    \item Create list of major commands and environments in \LaTeX\
\end{itemize}
Milestone 2: Parse \LaTeX\ file to SSML/MathML
\begin{itemize}
    \itemsep0em
    \item Understand how to use parser
    \item \LaTeX\ basic commands and environments will be parsed to a marked up file
\end{itemize}
Milestone 3: Implement Base \LaTeX\ Packages
\begin{itemize}
    \itemsep0em
    \item Implement package conversion
    \item Refine wavelengths (text analysis)
\end{itemize}
Milestone 4: Support Additional Packages
\begin{itemize}
    \itemsep0em
    \item Support additional \LaTeX\ packages in converter
    \item Continue to refine wavelengths (text analysis)
\end{itemize}
Milestone 5: Add UI Functionality/Additional Features
\begin{itemize}
    \itemsep0em
    \item Setup web browser for public usage
    \item Integrate with other applications such as Overleaf
\end{itemize}

\subsection{Resource Assignments}
Begin Comment  Resources include budgets, consumables like paint, and computing equipment
Begin Comment  like servers and development systems.  Assign these resources, if any, to the
Begin Comment  tasks, listed above. 
Begin Comment 
The only costly resources for our project will be the cloud-based TTS service and the server that will host our \TeX 2Speech application. Our top candidate for the TTS service, Amazon Polly, has a free trial period which extends beyond the development window for the initial programs release so this resource will not be a concern for the time being. The server hosting will be a bit more problematic, but will thankfully only be essential during the final phases of development. It will primarily be needed to test the UI and request handling portions of the project, so once we reach that stage a timesheet will have to be created to allot computation time.
\subsection{Individual Responsibilities}
Begin Comment  Who is responsible for which parts of the development effort.
Begin Comment 
As it stands, the project is currently undergoing a research period leading to future possibilities in task designation. Once a suitable foundation of background knowledge and experience in each aspect of the process is gained, we will then be able to appoint individual responsibilities based on acquired expertise.\section{Schedule}

\subsection{Gantt Chart}
% The initial task break-down schedule displayed graphically as a Gantt chart.
% (see https://www.fool.com/the-blueprint/gantt-chart/)
% Gnatt charts list the tasks, show the length of each task, and dependencies
% between tasks.  
\begin{figure}[h]
\includegraphics[width=14cm]{docs/vision_and_scope/Gantt.PNG}
\end{figure}

\subsection{Key Milestones}
% Milestones are points in the development where you have implemented some
% number of major features.  For senior project, you generally have five
% milestones: one at the end of 491, three in 492, and one in 493.
% In this section, divide the Major Features between the milestones.  You should
% front-load the schedule as much as possible.  That is, leave 10% of the work
% for the last milestone.
Milestone 1: Research and Implement Pipeline
\begin{itemize}
    \itemsep0em
    \item Research software to use for the project
    \item Setup TTS Web Service
    \item Find a placeholder for \LaTeX\ to SSML/MathML
    \item Find a parsing service for \LaTeX\ file
    \item Create simple web service for uploading files
    \item Create list of major commands and environments in \LaTeX\
\end{itemize}
Milestone 2: Parse \LaTeX\ file to SSML/MathML
\begin{itemize}
    \itemsep0em
    \item Understand how to use parser
    \item \LaTeX\ basic commands and environments will be parsed to a marked up file
\end{itemize}
Milestone 3: Implement Base \LaTeX\ Packages
\begin{itemize}
    \itemsep0em
    \item Implement package conversion
    \item Refine wavelengths (text analysis)
\end{itemize}
Milestone 4: Support Additional Packages
\begin{itemize}
    \itemsep0em
    \item Support additional \LaTeX\ packages in converter
    \item Continue to refine wavelengths (text analysis)
\end{itemize}
Milestone 5: Add UI Functionality/Additional Features
\begin{itemize}
    \itemsep0em
    \item Setup web browser for public usage
    \item Integrate with other applications such as Overleaf
\end{itemize}

\subsection{Resource Assignments}
% Resources include budgets, consumables like paint, and computing equipment
% like servers and development systems.  Assign these resources, if any, to the
% tasks, listed above. 
%
The only costly resources for our project will be the cloud-based TTS service and the server that will host our \TeX 2Speech application. Our top candidate for the TTS service, Amazon Polly, has a free trial period which extends beyond the development window for the initial programs release so this resource will not be a concern for the time being. The server hosting will be a bit more problematic, but will thankfully only be essential during the final phases of development. It will primarily be needed to test the UI and request handling portions of the project, so once we reach that stage a timesheet will have to be created to allot computation time.
\subsection{Individual Responsibilities}
% Who is responsible for which parts of the development effort.
%
As it stands, the project is currently undergoing a research period leading to future possibilities in task designation. Once a suitable foundation of background knowledge and experience in each aspect of the process is gained, we will then be able to appoint individual responsibilities based on acquired expertise.
\section{Deliverables}
Begin Comment  The purpose of this section is to formally list the major components that you
Begin Comment  will produce as part of this project.  

\subsection{Software/Hardware}
Begin Comment  List the software components and hardware, if any, that you will deliver as
Begin Comment  part of this project.  
\underline{\LaTeX\ to SSML Converter}: This is the largest and most fundamental piece of software we will be developing. It will contain all the rules for how different \LaTeX\ commands/syntax matches up with specifics of prosody and structure so proper pronunciation can be achieved. This piece of software will likely be further broken into several sub-modules to help aid in smooth development and managing the complexity of the task, but further research will be required to know where those distinctions will be drawn.\\
\newline
\underline{\TeX 2Speech Server}: This will be the request handling portion of the \TeX 2Speech system. It will manage interactions between users (their inputs and any files uploaded), the \LaTeX\ to SSML Converter and the TTS service upstream. While substantially simpler than the \LaTeX\ to SSML converter, its creation will still be involved and will determine how accessible our software is to use.

\subsection{Documentation}
Begin Comment  What user documentation will you deliver.
This project will include documentation regarding what features of \LaTeX\ \TeX 2Speech can support, including notable external packages and any limitations. It will specify the behavior upon encountering an error or unknown token, as well as acknowledgments of possible undefined edge cases. Along with this will be more basic documentation about how to navigate the program via its UI to upload \LaTeX\ files and get an audio output.

\subsection{Key Presentations}
Begin Comment  You will present your project at the end of 492 and 493.
\underline{End of CS 492}: We aim to present a program that converts \LaTeX\ to SSML. At this stage it will be invoked from a terminal. We will demonstrate the conversion of key \TeX\ functions to SSML using a predefined dictionary and analysis of document structure. We will then run this SSML through Amazon Polly to produce synthesized speech. \\
\newline
\underline{End of CS 493}: We will present document conversion using a more complete \TeX\ to SSML dictionary, and more sophisticated format analysis. Additionally, we will present a web application that allows users to easily convert LaTeX documents into WAV files.\section{Deliverables}
% The purpose of this section is to formally list the major components that you
% will produce as part of this project.  

\subsection{Software/Hardware}
% List the software components and hardware, if any, that you will deliver as
% part of this project.  
\underline{\LaTeX\ to SSML Converter}: This is the largest and most fundamental piece of software we will be developing. It will contain all the rules for how different \LaTeX\ commands/syntax matches up with specifics of prosody and structure so proper pronunciation can be achieved. This piece of software will likely be further broken into several sub-modules to help aid in smooth development and managing the complexity of the task, but further research will be required to know where those distinctions will be drawn.\\
\newline
\underline{\TeX 2Speech Server}: This will be the request handling portion of the \TeX 2Speech system. It will manage interactions between users (their inputs and any files uploaded), the \LaTeX\ to SSML Converter and the TTS service upstream. While substantially simpler than the \LaTeX\ to SSML converter, its creation will still be involved and will determine how accessible our software is to use.

\subsection{Documentation}
% What user documentation will you deliver.
This project will include documentation regarding what features of \LaTeX\ \TeX 2Speech can support, including notable external packages and any limitations. It will specify the behavior upon encountering an error or unknown token, as well as acknowledgments of possible undefined edge cases. Along with this will be more basic documentation about how to navigate the program via its UI to upload \LaTeX\ files and get an audio output.

\subsection{Key Presentations}
% You will present your project at the end of 492 and 493.
\underline{End of CS 492}: We aim to present a program that converts \LaTeX\ to SSML. At this stage it will be invoked from a terminal. We will demonstrate the conversion of key \TeX\ functions to SSML using a predefined dictionary and analysis of document structure. We will then run this SSML through Amazon Polly to produce synthesized speech. \\
\newline
\underline{End of CS 493}: We will present document conversion using a more complete \TeX\ to SSML dictionary, and more sophisticated format analysis. Additionally, we will present a web application that allows users to easily convert LaTeX documents into WAV files.

\end{document}
