\documentclass[10pt,reqno]{amsart}
\addtolength {\textwidth} {.5in}
%\addtolength {\oddsidemargin} {-.3in}
%\addtolength {\evensidemargin} {-.3in}
\usepackage{amssymb}
\usepackage{amsmath}
\usepackage{graphicx}
\theoremstyle{plain}
\newtheorem{Thm}{Theorem}
\newtheorem{Cor}[Thm]{Corollary}
\newtheorem{Main}{Main Theorem}
\newtheorem{Mres}{Main Result}
\newtheorem{\theMain}{}
\newtheorem{Lem}[Thm]{Lemma}
\newtheorem{Prob}{Problem}
\newtheorem{Prop}[Thm]{Proposition}
\theoremstyle{definition}
\newtheorem{Def}{Definition}
\theoremstyle{remark}
\newtheorem{notation}{Notation}
\renewcommand{\thenotation}{}

\def\biggdet#1{\biggl|#1\biggr|}
\long\def\comment#1{\textsf{\em[#1]}}
\renewcommand{\datename}{Version of}
\newcommand{\ntt}{\normalfont\ttfamily}
\newcommand{\pkg}[1]{{\protect\ntt#1}}
\DeclareMathOperator{\sgn }{sgn }
\DeclareMathOperator{\card }{card }
\DeclareMathOperator{\len }{len }
\DeclareMathOperator{\minor }{minor}
\DeclareMathOperator{\kr }{kr}
\parindent=0in
\parskip=10pt
\begin{document}

\begin{titlepage}

    \begin{center}
    \vspace*{1.325in}
    EDGE EFFECTS ON LOCAL STATISTICS IN LATTICE DIMERS:\\
    A STUDY OF THE AZTEC DIAMOND (FINITE CASE)\\
    \vspace{1.365626in}
    BY\\
    
    \vspace{.2in}
    HARALD HELFGOTT\\
    \vspace{1.11875in}
    THESIS\\
    \vspace{.2in} 
    Submitted in partial fulfillment of the requirements\\
    for the degree of Bachelor in Arts in Mathematics\\
    at Brandeis University, 1998\\
    \vspace{1.19750in}
    Waltham, Massachusetts
    \end{center}
    \end{titlepage}
    \address{Mathematics Department \\
      Brandeis University
      Waltham, MA 02254-9110}
    \email{hhelf@cs.brandeis.edu}
    %End topmatter
    \section{Introduction}

    A {\em tiling} of a checkerboard with dominoes is a way of putting dominoes
    on the board so that no square of the board is uncovered and no two dominoes
    overlap.  Given a local pattern (see figure \ref{fig:ex} for examples), a 
    location in the board, and the shape and size of the board, how many tilings
    of the board have the given pattern at the given location? (Alternatively,
    we can substitute ``bond'' for ``domino'' and ``particle'' for 
    ``square'', and ask for the probability of local patterns in a system
    of particles each of which bonds with exactly one of its neighbors.)
    
    Suppose that the squares of the board are very small compared to the
    board itself. For some board shapes, the probability of finding a pattern
    at a given location will be the same for almost all locations. This is
    the case for the square board. (See figures \ref{fig:g} to \ref{fig:i},
    where tiles are colored according to their direction and parity for
    the sake of clarity; see figure \ref{fig:chart} for the coloring scheme.)
    There are some boards, however, for which the probability does depend
    on the location. Consider, for example, the {\em Aztec diamond}, that is,
    the board whose boundary is a square tilted $45$ degrees 
    (figures \ref{fig:azb} and \ref{fig:azg}). 
    In random tilings of the Aztec diamond, we usually find brick-wall patterns 
    outside the inscribed circles, and more complicated behavior inside
    the circle. (See figures \ref{fig:b} to \ref{fig:f}.)
    
    The probabilities of local patterns in a rectangular board were computed
    recently \cite{Ken}. Until now, there was no other board for which
    the probabilities of all local patterns were known. Many experiments 
    and some important partial results \cite{CEP} had shown that, as already
    stated, the probabilities of patterns in the Aztec diamond depend on location.
    This qualitative difference between the Aztec diamond and the rectangular
    board made the former as worthy of analysis as the latter. 
    The main result of this work
    is an expression for the 
    probability of any local pattern in a random tiling of the Aztec diamond.
    The expression is a determinant of size proportional to the number of
    squares in the pattern, just like Kenyon's expression \cite{Ken} for
    the probabilities in the rectangural board,
     
    \begin{Mres}
    The probability of a pattern covering white squares $v_1,v_2,\dotsb v_k$
    and black squares $w_1,w_2,\dotsb w_k$ of
     an Aztec diamond of order $n$ is equal
    to the absolute value of
    \[\determinant{c(v_i,w_j)}_{i,j=1,2,\dotsb k}.\]
    The {\em coupling function} $c(v,w)$ at white square $v$ and black
    square $w$ is 
    \[2^{-n} \sum_{j=0}^{x_i-1} \kr(j,n,y_i-1) 
                    \kr({y\prime }_i - 1,n-1,n-(j+{x\prime }_i-x_i))
    \]
    for ${x\prime }_i > x_i$ and
    \[-2^{-n} \sum_{j=x_i}^n \kr(j,n,y_i-1) 
                 \kr({y\prime }_i-1,n-1,n-(j+{x\prime }_i-x_i))
    \]
    for ${x\prime }_i \leq x_i$, where $(x_i,y_i)$ and $({x\prime}_i,{y\prime}_i)$
    are the coordinates of $v$ and $w$, respectively, in the coordinate
    system in figure \ref{fig:coor}, and the {\em Krawtchouk polynomial}
    $\kr(a,b,c)$ is the coefficient of $x^a$ in $(1-x)^c\cdot (1+x)^{b-c}$.
    \end{Mres}

    Our line of attack is as follows.
\begin{enumerate}
\item Reduce the problem of finding probabilities of patterns to an 
enumerative problem;
\item Reduce the enumerative problem to a simpler one involving
Aztec diamonds with two holes rather than arbitrary even-area holes;
\item Compute the weighted number of tilings of an Aztec diamond with two holes.
\end{enumerate}

The first two steps involve known techniques, and were already considered
to be a plausible strategy by other researchers. The third step is new.

Henry Cohn is currently analyzing the case of the board with infinitely small
squares by approximating the sum of Krawtchouk polynomials in our main
result as an integral for $n\to \infty$. 
His results will be presented in a later, joint version
of this paper. 

\begin{figure}
\centering \includegraphics{locals.eps}
\caption{A few examples of local patterns} \label{fig:ex}
\end{figure}

\begin{figure}
	\centering \includegraphics[height=1in]{nomatter.eps}
	\caption{These two patterns
	have the same probability of being found in a random pattern
at any given place} \label{fig:nomatter}
\end{figure}

\begin{figure}
        \begin{minipage}[b]{0.5\linewidth}
                \centering \includegraphics[height=2in]{reg20.eps}
                \caption{Random tiling of square of side 20}\label{fig:g}   
        \end{minipage}
        \begin{minipage}[b]{0.5\linewidth}
                \centering \includegraphics[height=2in]{reg40.eps}
                \caption{Random tiling of square of side 40}\label{fig:h}
        \end{minipage}
\end{figure}

\begin{figure}
        \begin{minipage}[b]{0.5\linewidth}
                \centering \includegraphics[height=2.5in]{reg80.eps}
                \caption{Random tiling of square of side 80}\label{fig:i}   
        \end{minipage}
\end{figure}

\begin{figure}
\centering \includegraphics[height=2.5in]{chart.eps} 
\caption{Shading chart} \label{fig:chart}
\end{figure}

\begin{figure}
	\begin{minipage}[b]{0.5\linewidth}
		\centering \includegraphics[height=2in]{azb.eps}
	\caption{Aztec diamond of order $4$, as a board} \label{fig:azb}
	\end{minipage}
	\begin{minipage}[b]{0.5\linewidth}
		\centering \includegraphics[height=2in]{azg.eps}
	\caption{Aztec diamond of order $4$, as a graph} \label{fig:azg}
	\end{minipage}
\end{figure}

\end{document}




